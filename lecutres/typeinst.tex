\documentclass[runningheads,a4paper]{llncs}
\usepackage[T1]{fontenc}
\usepackage[polish]{babel} 
\usepackage{xunicode}
\usepackage{xltxtra} 
\usepackage{fontspec}
\usepackage{hyperref}
\hypersetup{    colorlinks,
    citecolor=black,
    filecolor=black,
    linkcolor=black,
    urlcolor=black}
\usepackage{graphicx}
\usepackage{wrapfig}
\usepackage{llncsdoc}

\begin{document}

\mainmatter 

\title{Inteligencja Obliczeniowa}
\titlerunning{Inteligencja Obliczeniowa}
\author{{}}
\institute{Uniwersytet Gdański -- Instytut Informatyki}

\authorrunning{Uniwersytet Gdański}
\tocauthor{{}}
\maketitle

\begin{abstract}
Notatki z wykładów z przedmiotu Inteligencja Obliczeniowa, Gdańsk, Jesień-Zima 2017.
\end{abstract}

\medskip
\begingroup
\let\clearpage\relax
\tableofcontents
\endgroup
\newpage

%
\section{Wykład 1 – 14.10.2017}
Inteligencja Obliczeniowa (CI) zajmuje się teorią rozwiązywania problemów, które nie są efektywnie algoytmiczne. Obejmuje wiele dziedzin np. sieci  neuronowe, algorytmy genetyczne i ewolucyjne, algorytmy mrówkowe i rojowe, systemy rozmyte, metody zgłębiania danych. Systemy korzystajace z CI przetwarzają i interpretują dane o różnorodnym charakterze, np. dane numer numeryczne, symboliczne.

\subsection{Algorytmy genetyczne}
Oparte na mechanizmach doboru naturalnego i dziedziczenia. Nie przechowują bezpośrednio  parametrów zadania, tylko ich zakodowaną postać, prowadzą poszukiwania wychodząć z pewnej populacji, nie z pojednyczego punktu, korzystają z funkcji przystosowania (celu). Osobniki towrzą populację – zawierają rozwiązanie, pupulacja zawiera osobników, istnieją operatory genetyczne i funkcje przypisania.

\subsubsection{Klasyczny algorytm genetyczny}
\begin{enumerate}
\item Inicjalizacja – wybór początkowej populacji,
\item Ocena przystosowania osobników w populacji,
\item Iteracja postępowania: 
\begin{enumerate}
\item selekcja osobników, 
\item zastosowanie operatorów genetycznych, 
\item ustalenie nowej populacji,
\end{enumerate}
\item Wypisanie najleszego osobnika.
\end{enumerate}

Operator mutacji odgrywa drugoplanową rolę w sotsunku do krzyżowania, które to występuje prawie zawsze, mutacja bardzo rzadko (z prawdopodbieństwem rzędu 0,01).

\subsubsection{Algorytm ewolucyjny}
Algorytm ewolucyjny jest pewnym uogólnieniem algoytmu genetycznego – nie musi być nieparzystą binarną opartą nie tylko genetycznie, ale dane do zrealizowania. Elementy określające algorytm genetyczny: sposób reprezentacji osobników, metoda zdefiniowania populacji początkowej, określenie funkcji przystosowania, wybór operatorów, określenie kryterium zakończenia.

\subsection{Przykłady}
\subsubsection{Problem Komiwojażera}
Jak określić funkcje przystosowania? Jak reprezenotwać osobniki?
Można uzyć tak zwanej reprezentacji ścieżkowej, czyli osobnik jest permutacją liczb od 1 do n; jak zdefiniować krzyżowanie i mutację? W reprezentacji prządkowej określa się tak zwany wzorzec, osobnik na i-tej pozycji może zawierać liczbę między 1 i n-i+1 np. dla n=7 wzorzec (1 2 3 4 5 6 7), osobnik jest listą (1 1 4 2 1 1 1) reprezentującą trasę 1-2-6-4-3-5-7.

\subsubsection{Szeregowanie Zadań}
Dany jest zbiór złożony z n zadań, ponadto dane są: czasy przetwarzania p\textsubscript{1},...,p\textsubscript{n}, d czas zakończenia, kary a\textsubscript{1},...,a\textsubscript{n} za wykonanie zadania zbyt wcześnie, kary b\textsubscript{1},...,b\textsubscript{n} za wykonanie zadania zbyt późno, jeśli ci jest czasem zakończenia i-tego zadania, to funkcję, która należy zminimalizować jest funkcja oceny.
%
\newpage
%
\section{Bibliografia}
\begin{enumerate}
\item L. Rutkowski – Metody i techniki sztucznej inteligencji, PWN,
\item J. Han – Data Mining. Concepts and Techniques, Morgan Kaufmann,
\item T. Morz – Eksploracja danych. Metody i algorytmy, PWN,
\item A.P. Engelbrecht – Computional Inteligence. An Itroduction, Wiley,
\item UCI Repository, http://archive.ics.uci.edu/ml,
\end{enumerate}

\end{document}
